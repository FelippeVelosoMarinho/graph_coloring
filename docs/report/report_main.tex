\documentclass[12pt,a4paper]{article}

% Pacotes básicos
\usepackage[utf8]{inputenc}     % Codificação do arquivo
\usepackage[T1]{fontenc}        % Acentos corretos
\usepackage[brazil]{babel}      % Português do Brasil
\usepackage{graphicx}           % Inclusão de imagens
\usepackage{float}              % Melhor controle de posição de figuras/tabelas
\usepackage{amsmath, amssymb}   % Símbolos matemáticos
\usepackage{hyperref}           % Links clicáveis
\usepackage{caption}            % Legendas personalizadas
\usepackage{cite}               % Gerenciamento de citações
\usepackage{listings}           % para formatar blocos de código
\usepackage{enumitem}           % para controlar listas
\usepackage{xcolor}             % necessário para cores no listings

% Configurações do listings
\lstset{
  language=Python,
  basicstyle=\ttfamily\small,
  numbers=left,
  numberstyle=\tiny,
  frame=single,
  breaklines=true,
  keywordstyle=\color{blue}\bfseries,
  stringstyle=\color{red},
  commentstyle=\color{green!60!black}\itshape,
  showstringspaces=false
}

\begin{document}

% ==============================
% CAPA
% ==============================
\begin{titlepage}
    \centering
    {\Large \textbf{Universidade Federal de Minas Gerais}}\\[0.3cm]
    {\large Engenharia de Sistemas}\\[2cm]
    
    {\Huge \textbf{Relatório do Trabalho Prático II - Coloração em Grafos}}\\[1.5cm]
    
    \textbf{Fundamentos de Inteligência Artificial}\\[0.5cm]
    \textbf{Professores:} Cristiano Castro e João Pedro Campos\\[1.5cm]
    
    \begin{flushleft}
        \textbf{Alunos:}\\
        Áquila Oliveira Souza --- 2021019327\\
        Arthur Jorge --- 2022055718\\
        Felippe Veloso Marinho --- 2021072260\\
        Jefferson Pereira de Souza --- 2022099049\\
        Josoé Santos Queiroz --- 2019026982
    \end{flushleft}
    
    \vfill
    {\large Belo Horizonte, MG}\\
    {\large \today}
\end{titlepage}

\clearpage
\tableofcontents
\clearpage

% ==============================
% INTRODUÇÃO
% ==============================
\section{Introdução}
A coloração de grafos é um problema clássico da teoria dos grafos com diversas aplicações práticas, como na alocação de frequências em redes sem fio, escalonamento de tarefas e planejamento de horários. O objetivo é atribuir cores aos vértices de um grafo de forma que vértices adjacentes não compartilhem a mesma cor, minimizando o número total de cores utilizadas.

Neste relatório, apresentamos uma solução para o problema de coloração de grafos, modelando-o formalmente como um grafo em que as arestas representam restrições binárias entre os vértices. 

Além disso, detalhamos as heurísticas utilizadas para resolver o problema, discutindo suas abordagens teóricas e implementações práticas. Também analisamos as decisões tomadas durante o desenvolvimento e os resultados obtidos.

O documento está organizado da seguinte forma: na Seção 2, apresentamos o problema de coloração de grafos; na Seção 3, discutimos as heurísticas teóricas; na Seção 4, detalhamos suas implementações; e, por fim, na Seção 5, apresentamos as conclusões e possíveis melhorias futuras.
\section{Problema da coloração em grafos}

\section{Heurísticas - Teoria}
Nessa sessão será definido como cada uma das heurísticas são.

\subsection{Random Walk (RW)}
O Random Walk é uma heurística de busca estocástica que explora o espaço de soluções movendo-se aleatoriamente entre as distribuições do grafo para encontrar uma coloração viável. 

A ideia principal é começar com uma coloração inicial (geralmente aleatória) e, em seguida, ajustar os conflitos, ou seja, resolver os casos em que vértices adjacentes possuem a mesma cor. Esse processo é repetido até que uma solução aceitável seja encontrada ou até que um critério de parada seja atingido.

\subsection{Best Improvement (BI)}
O Best Improvement é uma heurística que busca a melhor solução possível em cada iteração. O Best Improvement (BI) é uma heurística de busca local usada para melhorar uma solução já existente. 

A ideia é que, inicialmente, temos uma solução e, a partir dela, buscamos melhorá-la realizando pequenas alterações. Em vez de aceitar a primeira melhoria encontrada, o BI verifica todas as possíveis mudanças e escolhe aquela que oferece o maior benefício.

No contexto do problema de coloração de grafos, o objetivo é minimizar o número de conflitos (quando vértices vizinhos possuem a mesma cor) ou reduzir a quantidade de cores utilizadas para resolver o problema.

\subsection{First Improvement with Random Local Search (FI-RS)}

\subsection{First Improvement with Any Conflict (FI-AC)}

\subsection{Simulated Annealing (SA)}

\subsection{Algoritmo Genético (GA)}


\section{Heurísticas - Implementação}
\subsection{Random Walk (RW)}
Escolhe aleatoriamente uma variável e muda sua cor aleatoriamente.

\subsection{Best Improvement (BI)}
• Testa TODAS as mudanças possíveis (todos os vértices e cores).
• Escolhe a mudança que mais reduz conflitos.

\subsection{First Improvement with Random Local Search (FI-RS)}
• Escolhe uma variável aleatoriamente.
• Tenta uma cor aleatória para ela.
• Se melhorar, aceita.

\subsection{First Improvement with Any Conflict (FI-AC)}
• Escolhe uma variável que está em conflito.
• Tenta todas as cores possíveis.
• Escolhe a cor que mais reduz conflitos (best color para aquela variável).

\subsection{Simulated Annealing (SA)}
• Parecido com FI, mas aceita piores soluções com uma probabilidade que é
função do parâmetro de temperatura (que decai com o tempo) e da diferença
entre o número de conflitos entre a coloração (atribuição) atual e a nova
coloração.

\subsection{Algoritmo Genético (GA)}
• Evolui um conjunto de soluções candidatas, denominado população;
• A cada passo, as soluções atuais interagem entre si, através dos operadores
de recombinação (crossover) e mutação (mutation) para produzir uma nova
população.

\section{DSATUR}
\begin{itemize}
\item Explicação da heurística;
\item Implementação em código;
\item Decisões de implementação.
\end{itemize}


\section{Conclusão}
Apresentar as conclusões gerais do trabalho, destacando os principais aprendizados e possíveis melhorias futuras.

\section*{Referências}
\bibliographystyle{plain}
Inserir todas as referências utilizadas no mesmo formato (ABNT, APA ou Vancouver).  
Exemplo em ABNT:
\bibliography{references}

\end{document}