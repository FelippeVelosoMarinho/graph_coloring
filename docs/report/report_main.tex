\documentclass[12pt,a4paper]{article}

% Pacotes básicos
\usepackage[utf8]{inputenc}     % Codificação do arquivo
\usepackage[T1]{fontenc}        % Acentos corretos
\usepackage[brazil]{babel}      % Português do Brasil
\usepackage{graphicx}           % Inclusão de imagens
\usepackage{float}              % Melhor controle de posição de figuras/tabelas
\usepackage{amsmath, amssymb}   % Símbolos matemáticos
\usepackage{hyperref}           % Links clicáveis
\usepackage{caption}            % Legendas personalizadas
\usepackage{cite}               % Gerenciamento de citações
\usepackage{listings}           % para formatar blocos de código
\usepackage{enumitem}           % para controlar listas
\usepackage{xcolor}             % necessário para cores no listings

% Configurações do listings
\lstset{
  language=Python,
  basicstyle=\ttfamily\small,
  numbers=left,
  numberstyle=\tiny,
  frame=single,
  breaklines=true,
  keywordstyle=\color{blue}\bfseries,
  stringstyle=\color{red},
  commentstyle=\color{green!60!black}\itshape,
  showstringspaces=false
}

\begin{document}

% ==============================
% CAPA
% ==============================
\begin{titlepage}
    \centering
    {\Large \textbf{Universidade Federal de Minas Gerais}}\\[0.3cm]
    {\large Engenharia de Sistemas}\\[2cm]
    
    {\Huge \textbf{Relatório do Trabalho Prático II - Coloração em Grafos}}\\[1.5cm]
    
    \textbf{Fundamentos de Inteligência Artificial}\\[0.5cm]
    \textbf{Professores:} Cristiano Castro e João Pedro Campos\\[1.5cm]
    
    \begin{flushleft}
        \textbf{Alunos:}\\
        Áquila Oliveira Souza --- 2021019327\\
        Arthur Jorge --- 2022055718\\
        Felippe Veloso Marinho --- 2021072260\\
        Jefferson Pereira de Souza --- 2022099049\\
        Josoé Santos Queiroz --- 2019026982
    \end{flushleft}
    
    \vfill
    {\large Belo Horizonte, MG}\\
    {\large \today}
\end{titlepage}

\clearpage
\tableofcontents
\clearpage

% ==============================
% INTRODUÇÃO
% ==============================
\section{Introdução}
A coloração de grafos é um problema clássico da teoria dos grafos com diversas aplicações práticas, como na alocação de frequências em redes sem fio, escalonamento de tarefas e planejamento de horários. O objetivo é atribuir cores aos vértices de um grafo de forma que vértices adjacentes não compartilhem a mesma cor, minimizando o número total de cores utilizadas.

Neste relatório, apresentamos uma solução para o problema de coloração de grafos, modelando-o formalmente como um grafo em que as arestas representam restrições binárias entre os vértices. 

Além disso, detalhamos as heurísticas utilizadas para resolver o problema, discutindo suas abordagens teóricas e implementações práticas. Também analisamos as decisões tomadas durante o desenvolvimento e os resultados obtidos.

O documento está organizado da seguinte forma: na Seção 2, apresentamos o problema de coloração de grafos; na Seção 3, discutimos as heurísticas teóricas; na Seção 4, detalhamos suas implementações; e, por fim, na Seção 5, apresentamos as conclusões e possíveis melhorias futuras.
\section{Problema da Coloração de Grafos}
\label{sec:problema}

O problema da coloração de grafos consiste em associar uma cor a cada vértice de um grafo \( G = (V, E) \), de modo que dois vértices adjacentes \( u, v \in V \) não possuam a mesma cor. O desafio é minimizar o número total de cores utilizadas, conhecido como número cromático do grafo. Trata-se de um problema NP-difícil, o que motiva o uso de heurísticas e metaheurísticas para obtenção de soluções aproximadas em tempo viável.

\section{Heurísticas — Fundamentação Teórica}
\label{sec:heuristicas}

Nesta seção, são descritas as heurísticas empregadas para resolver o problema de coloração de grafos.

\subsection{Random Walk (RW)}

O \textit{Random Walk} é uma heurística de busca estocástica que explora o espaço de soluções movendo-se aleatoriamente entre colorações possíveis. O processo inicia-se com uma coloração aleatória e, em seguida, ajusta-se gradualmente os conflitos — isto é, os casos em que vértices adjacentes possuem a mesma cor. Esse procedimento é repetido até que se encontre uma solução viável ou que seja atingido um critério de parada.

\subsection{Best Improvement (BI)}

O \textit{Best Improvement} é uma heurística de busca local que busca a melhor melhoria possível em cada iteração. A partir de uma solução inicial, o algoritmo avalia todas as alterações viáveis e seleciona aquela que proporciona o maior ganho.  
No contexto da coloração de grafos, o objetivo é reduzir o número de conflitos (vértices adjacentes com a mesma cor) ou diminuir o total de cores utilizadas.

\subsection{First Improvement with Random Local Search (FI-RS)}

O \textit{First Improvement with Random Local Search} (FI-RS) é uma metaheurística de busca local que combina o \textit{Random Local Search} e o \textit{First Improvement}.  
O primeiro realiza pequenas alterações aleatórias em uma solução, enquanto o segundo aceita a primeira melhoria encontrada, sem necessidade de examinar todas as opções.  
Aplicado à coloração de grafos, o FI-RS busca uma coloração com o menor número de conflitos de forma eficiente, evitando uma exploração exaustiva do espaço de busca.

\subsection{First Improvement with Any Conflict (FI-AC)}

O \textit{First Improvement with Any Conflict} (FI-AC) é uma variação da heurística de busca local. A partir de uma solução inicial, possivelmente com conflitos, o algoritmo tenta melhorá-la iterativamente.  
A estratégia \textit{First Improvement} implica aceitar imediatamente o primeiro movimento que melhora a solução, sem buscar o melhor global. Já o termo \textit{Any Conflict} indica que qualquer vértice envolvido em conflito pode ser escolhido aleatoriamente para modificação.  
No problema de coloração de grafos, o FI-AC aprimora progressivamente a coloração até eliminar os conflitos entre vértices adjacentes.

\subsection{Simulated Annealing (SA)}

O \textit{Simulated Annealing} (SA) é uma metaheurística de busca estocástica inspirada no processo físico de recozimento térmico de metais. Nesse processo, o material é aquecido e resfriado lentamente para alcançar uma configuração estável de mínima energia.  
Analogamente, o SA aceita piores soluções no início (alta temperatura) e torna-se mais seletivo à medida que a “temperatura” diminui, buscando escapar de ótimos locais e aproximar-se de uma solução globalmente ótima.  
Na coloração de grafos, o SA visa reduzir gradualmente os conflitos de coloração, aceitando ocasionalmente soluções piores para explorar melhor o espaço de busca.

\subsection{Algoritmo Genético (GA)}
O \textit{Algoritmo Genético} (GA) é uma metaheurística inspirada nos princípios da evolução natural, proposta originalmente por Holland (1975). Ele baseia-se em mecanismos biológicos como seleção, cruzamento e mutação para evoluir uma população de soluções ao longo das gerações.

No contexto do problema de coloração de grafos, cada indivíduo da população representa uma coloração possível, onde os genes correspondem às cores atribuídas aos vértices. O processo evolutivo busca minimizar o número de conflitos e reduzir o total de cores utilizadas.


\section{Heurísticas - Implementação}
\subsection{Random Walk (RW)}
Escolhe aleatoriamente uma variável e muda sua cor aleatoriamente.

\subsection{Best Improvement (BI)}
• Testa TODAS as mudanças possíveis (todos os vértices e cores).
• Escolhe a mudança que mais reduz conflitos.

\subsection{First Improvement with Random Local Search (FI-RS)}
• Escolhe uma variável aleatoriamente.
• Tenta uma cor aleatória para ela.
• Se melhorar, aceita.

\subsection{First Improvement with Any Conflict (FI-AC)}
• Escolhe uma variável que está em conflito.
• Tenta todas as cores possíveis.
• Escolhe a cor que mais reduz conflitos (best color para aquela variável).

\subsection{Simulated Annealing (SA)}
• Parecido com FI, mas aceita piores soluções com uma probabilidade que é
função do parâmetro de temperatura (que decai com o tempo) e da diferença
entre o número de conflitos entre a coloração (atribuição) atual e a nova
coloração.

\subsection{Algoritmo Genético (GA)}
• Evolui um conjunto de soluções candidatas, denominado população;
• A cada passo, as soluções atuais interagem entre si, através dos operadores
de recombinação (crossover) e mutação (mutation) para produzir uma nova
população.

\section{DSATUR}
\begin{itemize}
\item Explicação da heurística;
\item Implementação em código;
\item Decisões de implementação.
\end{itemize}


\section{Conclusão}
Apresentar as conclusões gerais do trabalho, destacando os principais aprendizados e possíveis melhorias futuras.

\section*{Referências}
\bibliographystyle{plain}
Inserir todas as referências utilizadas no mesmo formato (ABNT, APA ou Vancouver).  
Exemplo em ABNT:
\bibliography{references}
\begin{thebibliography}{99}

\bibitem{qehaja2025}
QEHJA, B.; HAJRIZI, E.; ALI, M. A. A Hybrid Graph-Coloring and Metaheuristic Framework for Dynamic Wireless Sensor Networks. \textit{Preprints}, 2025. Disponível em: \url{https://www.preprints.org/manuscript/202507.0720/v1/download}. Acesso em: 26 out. 2025.

\bibitem{bihani2025}
BIHANI, O. A Heuristic for Graph Coloring Based on the Ising Model. \textit{Mathematics}, v. 13, n. 18, p. 2976, 2025. Disponível em: \url{https://www.mdpi.com/2227-7390/13/18/2976}. Acesso em: 26 out. 2025.

\bibitem{yakut2025}
YAKUT, S. A robust and efficient algorithm for graph coloring problem. \textit{Journal of King Saud University - Computer and Information Sciences}, 2025. Disponível em: \url{https://www.sciencedirect.com/science/article/pii/S1110866525000696}. Acesso em: 26 out. 2025.

\bibitem{dokeroglu2025}
DOKEROGLU, T.; BOGAZ, S.; KESKIN, M. An island-parallel ensemble metaheuristic algorithm for graph coloring problems. \textit{arXiv preprint arXiv:2504.15082}, 2025. Disponível em: \url{https://arxiv.org/html/2504.15082v1}. Acesso em: 26 out. 2025.


\end{thebibliography}


\end{document}